% vim: set spell spelllang=en tw=100 :

\documentclass[letterpaper]{article}

\usepackage[pass]{geometry}

\usepackage{ijcai16}
\usepackage{times}
\usepackage{complexity}
\usepackage{microtype}
\usepackage{amsmath}
\usepackage{amssymb}
\usepackage{amsthm}
\usepackage{cleveref}
\usepackage{tikz}
\usepackage{mathtools}
\usepackage{graphicx}

\usepackage{showframe}

\makeatletter\@ifpackageloaded{showframe}{
    % show lines down the middle columns too
    % http://tex.stackexchange.com/questions/16199/test-if-a-package-or-package-option-is-loaded
\newlength\Fcolumnseprule\setlength\Fcolumnseprule{0.4pt}\def\@outputdblcol{%
  \if@firstcolumn\global \@firstcolumnfalse \global \setbox\@leftcolumn \box\@outputbox
  \else
    \global \@firstcolumntrue
    \setbox\@outputbox \vbox{\hb@xt@\textwidth{\hb@xt@\columnwidth{\box\@leftcolumn \hss}%
    \vrule \@width\Fcolumnseprule\hfil{\normalcolor\vrule \@width\columnseprule}
    \hfil\vrule \@width\Fcolumnseprule\hb@xt@\columnwidth {\box\@outputbox \hss}}}\@combinedblfloats\@outputpage
    \begingroup\@dblfloatplacement\@startdblcolumn\@whilesw\if@fcolmade \fi{\@outputpage\@startdblcolumn}\endgroup
  \fi
}}{}
\makeatother

\usetikzlibrary{decorations, decorations.pathreplacing, calc, backgrounds}

\definecolor{uofgsandstone}{rgb}{0.321569, 0.278431, 0.231373}
\definecolor{uofglawn}{rgb}{0.517647, 0.741176, 0}
\definecolor{uofgcobalt}{rgb}{0, 0.615686, 0.92549}
\definecolor{uofgpumpkin}{rgb}{1.0, 0.72549, 0.282353}
\definecolor{uofgthistle}{rgb}{0.584314, 0.070588, 0.447059}

\newcommand{\citet}[1]{\citeauthor{#1} \shortcite{#1}}
\newcommand{\citep}[1]{\cite{#1}}

\theoremstyle{definition}
\newtheorem{proposition}{Proposition}

% cref style
\crefname{figure}{Figure}{Figures}
\Crefname{figure}{Figure}{Figures}

% http://tex.stackexchange.com/questions/22100/the-bar-and-overline-commands
\newcommand{\shortoverline}[1]{\mkern 1.5mu\overline{\mkern-1.5mu#1\mkern-1.5mu}\mkern 1.5mu}


% Maths operators
\newcommand{\nds}{\operatorname{nds}}
\newcommand{\lessnonind}[1]{\prescript{}{#1}{\rightarrowtail}\ }
\newcommand{\lessind}[1]{\prescript{}{#1}{\hookrightarrow}\ }

% to do comments
\usepackage{color}
\newcommand{\todo}[1]{{\color{red} {[}{#1}{]}}}


\title{Between Subgraph Isomorphism and Maximum Common Subgraph}

\author{Ruth Hoffmann \and Ciaran McCreesh\thanks{This work was supported by the Engineering and Physical Sciences
Research Council [grant number EP/K503058/1]} \and Craig Reilly \\
University of Glasgow, Glasgow, Scotland \\
ruth.hoffmann@glasgow.ac.uk \and \{c.mccreesh.1,c.reilly.2\}@research.gla.ac.uk}

\pdfinfo{
    /Title (Between Subgraph Isomorphism and Maximum Common Subgraph)
    /Author (Ruth Hoffmann and Ciaran McCreesh and Craig Reilly)
}

\begin{document}

\maketitle

\begin{abstract}
    When a small pattern graph does not occur inside a larger target graph, we can ask how to find
    ``as much of the pattern as possible'' inside the target graph. In general, this is maximum
    common subgraph problem, which is much more computationally challenging in practice than
    subgraph isomorphism. We present and evaluate a new constraint-style algorithm which lies
    between these two approaches, for cases when nearly all of the pattern is present.
\end{abstract}

\section{Introduction}

The subgraph isomorphism problem is to find a copy of a small \emph{pattern} graph inside a larger
\emph{target} graph. It comes in two forms: in the non-induced variant, edges must be mapped to
edges, but the target may have ``extra edges'', whilst in the induced variant, non-edges may only be
mapped to non-edges. When a pattern cannot be found, we may wish to be given a result which maps as
many vertices of the pattern into the target as possible. In the induced case, this is known as the
maximum common induced subgraph problem (we discuss the non-induced case below). However, although
recent subgraph isomorphism algorithms are comfortable working with graphs with thousands of
vertices, the state of the art for the maximum common subgraph problem becomes computationally
infeasible at only 35 vertices when working with unlabelled graphs \citep{CP2016MCSPaper}. This is
largely because strong inference, based upon the degrees of vertices
\citep{DBLP:journals/ai/Solnon10} and the distances or paths between them
\citep{DBLP:conf/cp/AudemardLMGP14,DBLP:conf/cp/McCreeshP15}, is possible with subgraph isomorphism,
but not maximum common subgraph, and so the state space in the former is much more restricted. In
this work we discuss an intermediate problem, where we must map all but $k$ vertices of the pattern
graph into the target. We show that if $k$ is reasonably small (say, between 1 and 5), then weakened
forms of degree and path based filterings are still effective in pruning the search space and
providing additional constraints respectively. We then show that combining these techniques does
lead to a practical algorithm which can scale to work with the families of graphs commonly used to
benchmark subgraph isomorphism algorithms: depending upon the benchmark family, in a substantial
portion of ; in many more cases, we can at least obtain an upper bound.

\subsection{Definitions and Notations}

A non-induced subgraph isomorphism from a graph $P$ to a graph $T$ is an injective mapping $P
\rightarrowtail T $ which maps adjacent vertices to adjacent vertices. An induced subgraph
isomorphism $P \hookrightarrow T$ additionally maps non-adjacent vertices to non-adjacent vertices.
We define a $k$-less-subgraph isomorphism from $P$ to $T$ to be a subgraph isomorphism from all but
$k$ vertices of $P$ to $T$; this may be non-induced or induced, written $P \lessnonind{k} T$ and $P
\lessind{k} T$ respectively.

A common induced subgraph of graphs $G$ and $H$ is a graph $P$, together with two induced subgraph
isomorphisms to $G$ and $H$; a maximum common induced subgraph is a common induced subgraph with as
many vertices as possible. An induced $k$-less subgraph isomorphism $P \lessind{k} T$ is equivalent
to a common induced subgraph of $P$ and $T$ with $|\operatorname{V}(P)| - k$ vertices.

If defined similarly, a maximum common non-induced subgraph would allow us to select every vertex in
the smaller of the two graphs, and none of the edges. It is therefore traditional to change the
objective to maximise the number of edges selected, rather than vertices, when a non-induced common
subgraph is sought---this problem is usually called the maximum common \emph{partial} subgraph
problem instead. However, this is not what we will be discussing in this paper: maximum common
subgraph problems are symmetric in their inputs, but when discussing the non-induced case we are
allowing extra edges only in the target graph, not in the pattern.

?? Draw an example.

\subsection{Constraint Models and Algorithms}

\subsection{Experimental Setup}

\section{Domain Filtering using Degrees}

A non-induced $k$-less-subgraph isomorphism from $P$ to $T$ is equivalent to a subgraph
isomorphism between $P$ and $T$ with $k$ extra universally-adjacent vertices. (However, we can do a
bit better algorithmically.) For induced $k$-less-subgraph isomorphisms, such an approach cannot
work.

Let $S$ and $T$ be two sequences. We define $S \preceq T$

\todo{Define $\preceq$, deg, nds}

\begin{proposition}
    Let $p$ be a vertex in $P$ and $t$ a vertex in $T$. For both non-induced and induced
    $k$-less-subgraph isomorphisms, if $p \prescript{}{k}{\mapsto} t$ then
    $\deg(p) - k \le \deg(t)$.
\end{proposition}
\begin{proof}
Let $p$ be a vertex in $P$, and $t$ a vertex in $T$, with $p\mapsto t$. Then by the definition of subgraph isomorphisms, $\deg(p) \le \deg(t)$. Let $P'$ be $P$ less $k$ vertices and $p$ a vertex in $P'$. Then
\[
\deg_{P}(p)-k \le \deg_{P'}(p) \le \deg_{P}(p) \le \deg(t).
\]
\end{proof}

\begin{proposition}
    More powerfully, if $p \prescript{}{k}{\mapsto} t$ then
    $(\nds(p)[k+1] - k, \nds(p)[k+2] - k, \ldots) \preceq \nds(t)$.
\end{proposition}

\begin{proof}
Let

\end{proof}

In \cref{figure:ids} we show that this is useful in practice.

\begin{figure}
    \includegraphics*{gen-graph-ids.pdf}
    \caption{We get domain filtering. Woohoo.}\label{figure:ids}
\end{figure}

\section{Filtering During Search Using Paths}

As well as reasoning about degrees, we can also reason about paths.

\todo{Define paths}

\begin{proposition}
    If $p \prescript{}{k}{\mapsto} t$ and $q \prescript{}{k}{\mapsto} u$ then
     $\operatorname{paths}(p, q, 2) \le \operatorname{paths}(t, u, 2) + k$.
\end{proposition}

This leads to the following:

\begin{proposition}
    For a graph $G$, let $G^{n, \ell}$ be the graph with vertex set $\operatorname{V}(G)$. The
    vertices $p$ and $q$ in $G^{n, \ell}$ are adjacent, if there are at least $n$ simple paths of
    length exactly $\ell$ between $p$ and $q$ in $G$. Then any $k$-less-subgraph isomorphism
    $P \lessnonind{k} T$ induces a new $k$-less-subgraph isomorphism
    $P^{n, 2}\lessnonind{k} T^{n - k, 2}$.
\end{proposition}

To allow for fast propagation, we create supplemental graphs for paths of length 2, looking at
counts of 1, 2, and 3 in the target graph (and so counts of $1 + k$ up to $3 + k$ in the pattern).
We then investigate whether this leads to new constraints being generated. By an assignment, we mean
considering mapping a pattern vertex $p$ to a target vertex $t$ (and not $\bot$) which does not
violate any loop constraints. An assignment pair is two assignments with distinct $p$ and distinct
$t$, which we say is permitted if it does not violate any adjacency constraint. We define the
permitted assignment pair ratio to be the proportion of assignment pairs which are permitted.
Finally, we scatter plot the permitted assignment pair ratio with and without supplemental
graphs\footnote{Because of the large sizes of the domains, we randomly sample one million pairs
rather than considering every pair. In some cases, we have nearly a thousand domains, each with nearly
ten thousand values---a complete quadratic calculation involving even a trivial arithmetic operation
on this would take many hours.}. For $k = 0$, we see many points above the $x-y$ diagonal, which
shows that for many instances, we are able to create a substantial number of new constraints at the
top of search; on the other hand, there are also points on the diagonal, which shows that sometimes
this technique provides no benefit. (Occasionally, points fall below the $x-y$ diagonal. This is
because the use of neighbourhood degree sequence reasoning on supplemental graphs can also lead to
increased domain filtering, which could in turn eliminate a higher proportion of forbidden than
permitted assignment pairs.) For $k = 1$ and $k = 2$, the proportion of points above the diagonal
diminishes, but we are still able to create new constraints for many instances. By the time $k = 3$,
most of the benefit is disappearing---although sometimes we are still able to make a difference, and
bear in mind that sometimes adding just one new constrained pair can vastly reduce the search space.

A closer inspection of the raw data shows that for $k = 3$, nearly all of the instances showing a
strong improvement are from the ``LV'' family of graphs.

\begin{figure*}
    \includegraphics*{gen-graph-constraints.pdf}\\[0.1cm]
    \includegraphics*{gen-graph-constraints-induced.pdf}
    \caption{We get new constraints. More woohoo.}\label{figure:constraints}
\end{figure*}

\section{A New Algorithm}

\section{Empirical Evaluation}

\begin{figure}
    \includegraphics*{gen-graph-runtimes.pdf}
    \caption{Look how amazing our stuff is.}\label{figure:runtimes}
\end{figure}

\subsection{Induced}

What proportion of instances are actually satisfiable as $k$ increases?

\begin{figure}
    \includegraphics*{gen-graph-which-k-by-family.pdf}
    \caption{There are satisfiable instances for small values of $k$. Yay.}\label{figure:which-k}
\end{figure}

\subsection{Non-Induced}

What proportion of instances are actually satisfiable as $k$ increases?

\subsection{Solving from the Top Down}

\begin{figure}
    \includegraphics*{gen-graph-versus-cp.pdf}
    \caption{A righteous spanking.}\label{figure:versus-cp}
\end{figure}

\section{Conclusion}

\section*{Acknowledgements}

We're not going to thank anyone, no, no.

\bibliographystyle{named}
\bibliography{paper}

\end{document}
